% 绘制图表.tex

\documentclass[a4paper, titlepage, hyperref, UTF8]{ctexart}
\usepackage{hyperref}
\usepackage{fancyhdr}
\usepackage{datetime}
\usepackage{scrtime}
\usepackage{amsmath}
\usepackage{mathtools}
\usepackage{bm}
\usepackage{commath}
\usepackage{dcolumn}
\usepackage{multirow}
\usepackage{graphicx}
\usepackage{lscape}
\hypersetup{
	colorlinks=false,
	pdfborder=0 0 0
}

\pagestyle{fancy}
\renewcommand{\footrulewidth}{0.4pt}
\fancyhf{}
\lhead{\leftmark}
\rhead{\thepage}
\cfoot{\thepage}



\title{绘制图表}
\author{张三}
\date{\today}

\begin{document}
\maketitle
\tableofcontents
\newpage

\section{LATEX中的表格}
按行列对齐的一组内容。二维延伸的特殊排版对象
\subsection{tabular和array}
两个环境:tabular,array
\begin{verbatim}

\begin{tabular}[垂直对齐]{列格式说明}
<表项>&<表项>&...&<表项>\\
\end{tabular}

\begin{array}[垂直对齐]{列格式说明}
<表项>&<表项>&...&<表项>\\
\end{array}

[垂直对齐]参数可以是:
t表格顶部对齐
b表格底部对齐
默认垂直居中

\end{verbatim}

\begin{tabular}{lcr}
left & center & right \\
本列左对齐 & 本列居中 & 本列右对齐 \\
\end{tabular}

\begin{tabular}{ll}
\bfseries 功能 & \bfseries 环境 \\
表格 & \sffamily \itshape tabular \\
对齐 & \ttfamily tabbing
\end{tabular}

\begin{tabular}{|l|}
\hline
上 \\
\hline
中 \\
\hline
下 \\
\hline
\end{tabular}
\qquad
输入与输出有关系$y=x^2$

\begin{tabular}{|c|rrr|p{4em}|}
\hline
姓名 & 语文 & 数学 & 外语 & 备注 \\
\hline
张三 & 87 & 100 & 93 & 优秀 \\
李四 & 75 & 63 & \emph{52} & 补考另行通知  \\
王小二 & 80 & 82 & 78 & \\
\hline
\end{tabular}

\begin{tabular}{|c|r@{.}l|}
\hline
收入 & 12345 & 6 \\ \hline
支出 & 765 & 43 \\ \hline
结余 & 11580 & 17 \\ \hline
\end{tabular}

\begin{tabular}{|c|*{3}{r@{.}l|}}
\hline
收入 & 12345&6 & 5000&0 & 1020&55\\ \hline
支出 & 765&43  & 5120&5 & 98760&0  \\ \hline
结余 & 11580&17 & -120&5 & -97739&45 \\ \hline
\end{tabular}




\[
\begin{pmatrix}
\begin{array}{ccc|c}
a_{11} & a_{12} & a_{13} & b_{1} \\
a_{21} & a_{22} & a_{23} & b_{2} \\
a_{31} & a_{32} & a_{33} & b_{3} 
\end{array}
\end{pmatrix}
\]

\subsection{表格单元的合并于分割}
\verb|\multicolumn{项数}{新列格式}{内容}|命令可用于将一行中几个不同的表项合并为一项
例如:
\begin{tabular}{|r|r|}
\firsthline
\multicolumn{2}{|c|}{成绩} \\ \hline
语文 & 数学 \\ \hline
87 & 100 \\ 
\lasthline 
\end{tabular}

\begin{tabular}{|r|r|}
\firsthline
\multicolumn{1}{|c|}{输入} & \multicolumn{1}{c|}{输出} \\ \hline
1 & 1 \\ \hline
5 & 25 \\ \hline
15 & 225 \\ 
\lasthline
\end{tabular}

\begin{tabular}{|c|r|r|}
\firsthline
& \multicolumn{2}{c|}{成绩} \\ \cline{2-3}
姓名 & 语文 & 数学 \\ \hline
张三 & 87 & 100\\
\lasthline
\end{tabular}

\begin{tabular}{|c|}
\firsthline
1 \\ \hline
1 \vline 2 \\ \hline
1 \vline 2 \vline 3 \\
\lasthline
\end{tabular}

\begin{tabular}{|c|}
\firsthline
1 \\ \hline
\begin{tabular}{c|c} 1 & 2 \end{tabular} \\ \hline
\begin{tabular}{@{}c|c|c@{}} 1 & 2 & 3 \end{tabular} \\ 
\lasthline
\end{tabular}

\begin{tabular}{|c|r|r|}
\firsthline
\multirow{2}*{姓名} & \multicolumn{2}{c|}{成绩} \\ \cline{2-3}
& 语文 & 数学 \\ \hline
张三 & 87 & 100 \\ 
\lasthline
\end{tabular}



\subsection{定宽表格与tabularx}
\subsection{长表格与longtable}
\subsection{三表线与表线控制}
\subsection{array宏包与列格式控制}
\subsection{定界符与子矩阵}

\section{插图与变换}
\LaTeX 中使用三类图形:
\begin{itemize}
\item 使用\TeX 基本的命令  
\item 使用特殊的字体拼接组合
\item 使用\TeX 的扩充接口\verb"\special" 命令或新引擎的功能
\end{itemize}

\subsection{graphicx与插图}
\begin{verbatim}
通过\usepackage{graphics}, \usepackage{graphicx}
\includegraphics[keyvals]{imagefile}
\end{verbatim}

\includegraphics[width=2em]{o59gvl}
\includegraphics[height=1cm]{o59gvl}
\includegraphics[scale=0.5]{o59gvl}

\includegraphics[scale=0.5,angle=90]{28p95m}
\includegraphics[scale=0.5,angle=90,origin=c]{28p95m}

基线\rule{2cm}{0.4pt}
\includegraphics[height=1cm]{28p95m}
\includegraphics[angle=90,origin=l,height=1cm]{28p95m}
\includegraphics[angle=90,origin=b,totalheight=1cm]{28p95m}
\includegraphics[angle=90,origin=lb,totalheight=1cm]{28p95m}

图片格式:PDF矢量图、PNG无损压缩像素图、JEPG有损压缩像素图





\subsection{几何变换}
\subsection{页面旋转}
\verb|\begin{landscape} content... \end{landscape}|




\section{浮动体与标题控制}
\subsection{浮动体}
两种浮动体环境:\verb|figure table|分别用于图和表的排版
\begin{verbatim}
\begin{figure}[允许位置]默认为tbp
content...
\end{figure}

\begin{table}[任意位置]
content...
\end{table}

位置:
h(here):此处
t(top):页顶
b(bottom):页底
p(page):独立一页(浮动页,文本页)
在table环境中放置tabular生成的表格,在figure放置\includegraphics[keyvals]{imagefile}
命令插入的图片.还在前使用\centering命令让图表居中

\caption{text}
\caption[short text]{text}
可以为图表加标题
\end{verbatim}

\begin{figure}[htbp]
	\centering
	\includegraphics{28p95m}
\end{figure}

\begin{table}
	\centering
	\begin{tabular}{|c|c|}
	\firsthline
	图形 & \verb|figure|环境\\ \hline
	表格 & \verb|table|环境\\
	\lasthline
	\end{tabular}
\end{table}

\begin{figure}
\centering
\includegraphics{28p95m}
\caption[风景]{\TeX{} 的吉祥物———风景}\label{fig-28p95m}
\end{figure}



\subsection{标题控制与caption宏包}
\subsection{并排与子图表}
\subsection{浮动控制与float包}
\subsection{文字绕排}

\section{使用彩色}
\subsection{彩色表格}

\section{绘图语言}
\subsection{XY-pic与图标交换}
\subsection{PSTricks与TikZ简介}
\subsection{METAPOST与Asymptote简介}




\end{document}