% coding: UTF-8
% paragraph.tex

\documentclass{ctexart}

\usepackage{fontspec}
\usepackage{theorem} % 定理类宏包
\usepackage{listings}


\lstset{ % 整体设置
    columns = flexible,
    numbers = left,
    escapechar = ',
    numberstyle = \footnotesize,
    basicstyle = \sffamily,
    keywordstyle = \bfseries,
    commentstyle = \rmfamily\itshape,
    stringstyle = \ttfamily
}

\renewcommand{\tablename}{目录}

\newcounter{mycnt}  % 定义一个计数器
\setcounter{mycnt}{0} % 设置计数器的值
\addtocounter{mycnt}{10} % 计数器值增减

\theoremstyle{marginbreak}
\newtheorem{thm}{定理} 
% 在编书时 \newtheorem{lemma}{引理}[chapter] % [chapter]可选参数可以按章计数
\newtheorem{prop}[mycnt]{命题}




\title{Paragraph}
\author{Umbrella}
\date{\today}
\bibliographystyle{plain}

\begin{document}
\maketitle
\tableofcontents	
\section{正文段落}

段落的相关命令

% 中文默认有缩进,英文默认无缩进,但是英文可以indentfirst宏包产生默认缩进
% \parindent控制缩进 \indent控制缩进 \noindent取消缩进
% 段与段的垂直距离由\parskip控制
% 段落的对齐方式默认两端均匀对齐,\raggedright段落左对齐,\raggedleft右对齐
% \centering居中对齐
% 文字对齐环境,flushleft左对齐,flushright右对齐,center居中环境



\section{文本环境}

% 三种: 引用,诗歌, 摘要

% 引用环境
\begin{quote}
引用环境

首行无缩进,\par
用于小段
\end{quote}

\begin{quotation}
引用环境

首行有缩进,\par
用于多段
\end{quotation}

% 摘要环境
% 类似于quote,但是添加了一个标题,并可以通过\abstractname{name}定义
\begin{abstract}
本书讲解\LaTeX{} 的使用
\end{abstract}

\section{列表环境}

% 三种列表环境,编号的enumerate、不编号的itemize、和关键字的description
% 列表内使用\item 命令开始一个列表项,带一个可选参数表示手动编号或关键字
% 三种环境最多可以嵌套四层

% enumerate使用数字自动编号
% 有一个计数器(counter)控制编号,四层嵌套使用四个计数器: enumi,enumii,enumiii,enumiv
% \the计数器名输出计数器值

\begin{enumerate}
	\item 中文
	\begin{enumerate}
		\item 古代汉语
		\begin{enumerate}
			\item 口语
			\begin{enumerate}
				\item 普通话
				\item 方言
			\end{enumerate}
			\item 书面语
		\end{enumerate}
		\item 现代汉语
	\end{enumerate}
	\item English
	\item Francais
\end{enumerate}

% itemize不编号,加一个符号
\begin{itemize}
	\item 中文
	\item English
	\item Francais
\end{itemize}

% description使用\item 的可选参数,并作为关键字加粗标记
\begin{description}
	\item[中文] 中国的语言文字
	\item[English] The language of England
	\item[Francias] La langue de France
\end{description}

\section{定理类环境}	

% 产生一个标题、编号和特定格式的文本
% 在导言区定义声明定理环境
% 定理类环境计数器就是定理名,会随定理个数增加自增

\begin{thm}[勾股定理]
这是一个定理
\end{thm}
\begin{thm}
第二个定理
\end{thm}
\begin{prop}
直角三角形的斜边大于直角边
\end{prop}

% 定理类宏包
% 预定义格式 \theoremstyle{style}
% plain	       默认格式 
% break		   定理头换行
% marginbreak  编号在页边,定理头换行
% changebreak  定理头编号在前文字在后,换行
% change	   定理头编号在前文字在后,不换行
% margin 	   编号在页边,定理头换行


\section{抄录和代码环境}


% 一小段文字,使用\verb(*)|text|,加*使空格可见,
% 使用|text|,"text",!text!等 <符号>文本<符号>将需展示文本指出
\verb|"\LaTeX \& \Tex"|\par
\verb"\LaTeX \& \Tex"\par
\verb!\/}{#$%&~!\par
\verb*|\LaTeX \& \Tex|

% 大段使用verbatim环境
\begin{verbatim}
#include <iostream>
using namespace std;
int main() {
    cout << "Hello World!" << endl;
}
\end{verbatim}

\subsection{程序代码与listings}

% 程序代码的语法高亮
% 使用listings宏包,使用lstlisting环境
% 在可选参数中选择语言
% listings对中文支持不好,需使用逃逸字符处理, 
% 设置逃逸字符escapechar = ', 用''括起的字符
\begin{lstlisting}[language=C++]
/* hello.cpp */
#include <iostream>
using namespace std;
int main() {
    cout << "Hello World!" << endl;	// '输出'
}
\end{lstlisting}


\section{tabbing环境}

% 用来排版制表位
% 行与行用\\分离,使用\=命令设置制表位, \>跳到下一个前面已设置的制表位
% 可以用来制无线表格
% 命令有\', \', \<, \>, \+, \-, \=, \pushtabs, \poptabs


\section{脚注与边注}

% \footnote{text}产生脚注内容, 大小为\footnotesize
例如\footnote{脚注}

知识图谱\footnote[1]{人工智能}

区块链\footnote{比特币}






\end{document}