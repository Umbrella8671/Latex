% 自动化.tex

\documentclass[a4paper, titlepage, hyperref, UTF8]{article}	% 英文格式的目录在左边 
% \documentclass[titlepage]{ctexart} % 中文会自动将目录设在中间
\usepackage{ctex}
\usepackage{fancyhdr}
\usepackage{hyperref}

\hypersetup {
    colorlinks = true,
    allcolors = black,
    bookmarks = true,
    bookmarksopen = true,
    bookmarksnumbered = true,
    pdfborder = 0 0 0, 
    pdfstartview = Fit,
    pdftitle = 自动化,
    pdfauthor = 张三,
    pdfsubject = 自动化介绍,
}


\CTEXoptions[today=old]




\renewcommand \tablename{目录}
% 定义页眉页脚
\pagestyle{fancy}
\fancyhf{}
\renewcommand{\footrulewidth}{0.4pt}
\cfoot{\thepage}
\lhead{\rightmark}
\rhead{\thepage}
\fancyhead[LE]{\rightmark}

\title{自动化工具}
\author{无名氏}
\date{\today}
\bibliographystyle{plain}

\begin{document}
\addcontentsline{toc}{section}{自动化工具}
\maketitle
\tableofcontents
\section{目录}
\subsection{目录和图表工具}

\subsection{控制目录内容}
可以手工制作目录

\addcontentsline{toc}{section}{\numberline {5}手工目录}{2}
\addcontentsline{toc}{subsection}{\numberline {5.1}手工子目录}
\subsection{定制目录格式}



\section{交叉引用}
可以通过符号书签引用文档中某个对象的编码、页码或标题等信息\\
自动建立书签


\subsection{标签与引用}
使用分两步: 定义标签,引用标签



\subsection{更多交叉引用}
使用宏包
\subsection{电子文档与超链接}
使用hyperef宏包\\
\verb|\usepackage{hyperref}|
\begin{verbatim}
\hypersetup {
    colorlinks
    bookmarks
    bookmarksopen
    bookmarksnumbered
    pdfborder
    pdfpagemode
    pdfstartview
    pdftitle
    pdfauthor
    pdfsubject
    pdfkeywords
}
\end{verbatim}


\verb|\url{text}用来输出URL地址,也有超链接功能|
例如:\url{www.baidu.com}

\verb|\nolinkurl{URL}输出的URL,不具备超链接功能|
例如:\nolinkurl{www.baidu.com}

\verb|\path命令可以用来排版文件路径, 格式和\url差不多|

\verb|\href{URL}{text},可使文字产生指向URL地址的超链接效果|\par
例如:\href{www.baidu.com}{百度}

\verb|\href[options]{text},可以让文字指向标签|\par

\verb|\phantomsection产生一个超链接点,用于手工添加的目录|\par
\verb|\pdfbookmark[层次]{书签文字}{链接点名称}手工添加PDF书签|


\section{BIBTEX与文献库数据}
通过外部工具BIBTEX,根据文章内容从一个文献库抽取、\\
整理和排版文献


\subsection{BIBTEX基础}
文献数据库是以.bib结尾的文本文件
内容大致为:
\begin{verbatim}
@BOOK {
   mittelbach2004,
   title = {The {\LateX} Companion},
   publisher = {Addison-Wesley},
   year = {2004},
   author = {Frank Mittelbach and Mickel Goossens},
   seriers = {Tools and Techniques for Computer Typesetting},
   address = {Boston},
   edition = {Second}
}

三件事:
1. 使用\bibliographystyle{style}于导言区设定参考文献格式
有plain、alpha、abbrv等
2. 在正文中用标签使用\cite命令引用需要的文献,或\nocite指明
不引用但仍需列出的文献标签
3. 使用\bibliography指出要使用的文献数据库
\end{verbatim}


\subsection{JabRef与文献数据库管理}
\subsection{用natbib定制文献格式}
\subsection{更多的文献格式}
\subsection{文献列表的底层命令}


\section{Makeindex索引}
\subsection{制作索引}

\subsection{定制索引格式}
\subsection{词汇表及其他}


\bibliography{自动化}

\end{document}
