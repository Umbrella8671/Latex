% 玩转数学公式.tex

\documentclass[titlepage, hyperref, UTF8]{ctexart}
\usepackage{hyperref}
\usepackage{fancyhdr}
\usepackage{amsmath}
\usepackage{mathtools}
\usepackage{mathdots}
\usepackage{bm}
\usepackage{commath}


\hypersetup{
	colorlinks=false,
	pdfborder=0 0 0
}

\pagestyle{fancy}
\renewcommand{\footrulewidth}{0.4pt}
\fancyhf{}
\rhead{\thepage}
\lhead{\leftmark}
\cfoot{\thepage}




\title{玩转数学公式}
\author{张三}
\date{\today}


\begin{document}
\maketitle
\tableofcontents
\newpage

\section{数学模式概说}
使用宏包\verb|\usepackage{amsmath}|\par
\TeX{} 有两种数学模式:\\
一种是夹杂在行文段落中的公式,称为行内(inline)或正文(in-text)数学公式,用一对\verb|$$ 或\(\) 或\begin{math}\end{math}|来表示;\\
另一种是单独占据整行居中展示出来,称为显示(displayed)公式(或行间、列表)。\\
可以用\verb|$$contents$$, \[contents\], displaymath环境|\\
带编号的数学环境用equation


\section{数学结构}
\subsection{上标与下标}
上标:\^{}\par
下标:\_{}

$10^n$, $A_i$, $A_{ij} = 2^{i+j}$

可以同时使用,也可以嵌套,但是嵌套时外层分组

$A_i^k = B^k_i$ \qquad 
$K_{n_i} = k_{2^i}
		 = 2^{2^i}$\qquad
$3^{3^{3^{\cdot^{\cdot^{\cdot^3}}}}}$

撇号$\prime$是特殊上标$'$,可以连续使用,特殊上标不可连续使用,不可与普通上标直接混用

$a = a'$, $b_0' = b_0''$,
${c^\prime}^2 = (c')^2$

角度:$\circ$,
$A = 90^\circ$

或定义一个更明显的命令:\verb|\newcommand{\degree}{^\circ}|
\newcommand{\degree}{^\circ}

$90\degree$

四个命令
\begin{verbatim}
    \limits
    \nolimits
    \prescript{sup}{sub}{arg}
    \sideset{left}{right}{symbol}
\end{verbatim}

\verb|$\limits$限制上下标在正上下方|

\[
\iiint\limits_D \mathrm{d}f =
\max\nolimits_D g
\]
$\sum\limits_{i=0}^n A_i$不如用
$\sum_{i=0}^n A_i$更适合文本段落

左上坐下加角标:\verb|${}_m^nH$|得到
${}_m^nH$但是间距有问题,可以用mathtools宏包

使用\verb|\prescript{sang}{sub}{arg}|例如:\verb|$\prescript{n}{m}{H} < L$|得到$\prescript{n}{m}{H} < L$

\[
\sideset{_a^b}{_c^d}\sum_{i=0}^n A_i
= \sideset{}{'} \prod_k f_i
\]

\subsection{上下划线与花括号}
上下划线:\verb|\overline \underline|, 可以嵌套
$\overline{a+b} = 
  \overline a + \overline b$\\
$\underline a = (a_0, a_1, a_2, \dots)$

$\overline{\underline{\underline{a}} + \overline{b}^2}
- c^{\underline n}$

还有上下加箭头矢量:
\begin{verbatim}
$\overleftarrow{text}$
$\overrightarrow{text}$
$\overleftrightarrow{argument}$
$\underleftarrow{argument}$
$\underrightarrow{argument}$
$\underleftrightarrow{argument}$
\end{verbatim}
例如:\\
$\overleftarrow{a+b}$\\
$\overrightarrow{a+b}$\\
$\overleftrightarrow{a+b}$\\
$\underleftarrow{a+b}$\\
$\underrightarrow{a+b}$\\
$\underleftrightarrow{a+b}$\\

花括号(加标注):\verb|$\overbrace{text}$ $\underbrace{text}$|
$\overbrace{a+b+c}=\underbrace{1+2+3}$
\[
(\overbrace{a_0, a_1, \dots, a_n}^{\text{共$n+1$项}})
= (\underbrace{0, 0, \dots, 0}_{n}, 1)
\]

方括号:\verb|$\overbracket{text}$, $\underbracket{arg}$|

\[
a+ \rlap{$\overbrace{\phantom{b+c+d}}^m$} b+ \underbrace{c+d+e}_n +f
\]
\subsection{分式}
分式(fraction):\verb|\frac<分子><分母>|
例如:$\frac12 + \frac1a=\frac{2+a}{2a}$

指定较大或较小的分式用:\verb|$\dfrac, \tfrac$|
例如:
\[
\tfrac12f(x) = 
\frac1{\dfrac1a + \dfrac1b + c}
\]

连分式(continued fraction):\verb|\cfrac[对齐方式(l,c,r)]|
例如:
\[
\cfrac{1}{1+\cfrac2{1+\cfrac3{1+x}}}=
\cfrac[r]{1}{1+\cfrac2{1+\cfrac[l]{3}{1+x}}}
\]

上下两半分式:\verb|\binom|
例如:
\[
(a+b)^2=\binom 20 a^2+\binom21 ab
+\binom22b^2
\]

\subsection{根式}
根式:\verb|$\sqrt[开方次数]{开方数}$|
例如:$\sqrt 4=\sqrt[3]{8}=2$
\subsection{矩阵}
矩阵环境6个:matrix, bmatrix, vmatrix,
pmatrix, Bmatrix, Vmatrix, 以及小型矩阵smallmatrix;可以嵌套

不同的列用\&分隔,行用\textbackslash\textbackslash 分割

\[
A = \begin{pmatrix}
a_{11}&a_{12}&a_{12}\\
0&a_{22}&a_{23}\\
0&0&a_{33}
\end{pmatrix}
\]
矩阵中的省略号:\verb|\dot, \dots, \vdots, \ddots|
\[
A = \begin{bmatrix}
a_{11}&\dots&a_{1n}\\
&\ddots&\vdots\\
0& &a_{nn}
\end{bmatrix}_{n\times n}
\]
\verb|\hdotsfor{列数}|多列省略号
\[
\begin{Vmatrix}
1& \frac12& \dots& \frac1n\\
\hdotsfor{4}\\
m& \frac m2& \dots& \frac mn
\end{Vmatrix}
\]
\[
\begin{Bmatrix}
\begin{matrix} 1& 0\\ 0& 1 \end{matrix}
& \text{\Large 0}\\
\text{\Large 0}& 
\begin{matrix} 1& 0\\ 0& -1 \end{matrix}
\end{Bmatrix}
\]

\verb|\substack 和 subarray环境|
\[
\sum_{\substack{0<i<n\\ 0<j<i}} A_{ij}
\]
\[
\sum_{
\begin{subarray}{c}
i<10\\ j<100\\ k<1000
\end{subarray}}
X(i,j,k)
\]
\section{符号与类型}
数学符号分:普通符号,巨算符,二元运算符,关系符,开符号,比符号,标点和变量族(一般是字母)
\subsection{字母表与普通符号}
键盘获取的字母

数学字体
\begin{verbatim}
$\mathnormal{text}$
$\mathrm{text}$
$\mathit{text}$
$\mathbf{text}$
$\mathsf{text}$
$\mathtt{text}$
$\mathcal{text}$
\end{verbatim}

小写希腊字母
\begin{verbatim}
$\alpha$	$\beta$		$\gamma$	$\delta$
$\epsilon$	$\zeta$		$\eta$		$\theta$
$\iota$		$\kappa$	$\lambda$	$\mu$
$\nu$		$\xi$		$\pi$		$\rho$
$\sigma$	$\tau$		$\upsilon$	$\phi$
$\chi$		$\psi$		$\omega$	
\end{verbatim}

大写希腊字母:
\begin{verbatim}
$\Gamma$	$\Delta$	$\Theta$	$\Lambda$
$\Xi$		$\Pi$		$\Sigma$	$\Upsilon$
$\Phi$		$\Psi$		$\Omega$
\end{verbatim}

粗体数学符号:\verb|\boldmath|
{\boldmath $a^2$}

\newcommand{\mi}{\mathrm i}
\newcommand{\me}{\mathrm e}
$\me^{\pi \mathrm i} + 1 = 0$

\subsection{数学算子}
数学算子(math operator)分两种:第一类是类似求和号$\sum$和积分号$\int$
等大小可变的,称为巨算符。\\
第二类算子是文字名称的算子。

巨算符(可以带上下标)\footnote{积分号在角标位置,其他默认在上下方}:
\begin{verbatim}
\sum    \prod    \coprod
\int    \oint    
\bigcup \biguplus \bigsqcup
\bigvee \bigwedge \bigcap
\bigodot \bigoplus \bigotimes
\iint   \iiint     \iiiint
\idotsint
\end{verbatim}
\[
\sum    \prod    \coprod
\int    \oint    
\bigcup \biguplus \bigsqcup
\bigvee \bigwedge \bigcap
\bigodot \bigoplus \bigotimes
\iint   \iiint     \iiiint
\idotsint
\]

不带上下限的数学算子名\footnote{做数学函数名}:
\begin{verbatim}
\log    \lg    \ln    \sin    \arcsin
\cos    \arccos\tan   \arctan \cot
\sinh   \cosh  \tanh  \coth   \sec
\css    \arg   \ker   \dim    \hom
\exp    \deg
\end{verbatim}

带上下限的数学算子名\footnote{用法和求和、积分号类似}:
\begin{verbatim}
\lim    \limsup    \liminf    \max
\min    \sup      \inf       \det
\Pr     \gcd
\end{verbatim}

\begin{equation}
\oint_C \kappa_\mathrm g \dif s + \iint_D K \dif\sigma = 2\pi - \sum_{i=1}^n \alpha_i
\end{equation}
\begin{equation}
\lim_{x\to -\infty} f(x) = \infty
\end{equation}

\subsection{二元运算符与关系符}
二元运算符有普通二元运算符及其否定,关系符,标准字体加上AMS字符

\subsection{括号与定界符}
园、方、花、尖括号等。
左边的括号是开符号,右边的括号是闭符号\\
\verb|\lbrace, \rbrace|花括号\\
\verb|\langle, \rangle|尖括号\\
\verb|\lfloor, \rfloor|向下取整\\
\verb|\lceil, \rceil|向上取整

\subsection{标点}
数学标点:
, \quad ; \quad ! \quad ? \quad $\colon$


\section{多行公式}
一行写不下的公式
\subsection{罗列多行公式}
把公式罗列在一起,是基本的产生多行公式的方法。有eqnarry环境\\
equation环境中\textbackslash \textbackslash 不产生换行\\
gather环境则可以换行,公式居中\\
align环境允许公式按等号或其他关系符对齐,关系符前加\&表示对齐,还允许排列多列对齐公式,列用\&分开

\begin{gather}
3+5=5+3=8  \\
3\times 5 = 5\times 3 \notag % 阻止指定的行不编号 
\end{gather}

\begin{align}
	x &= t+\cos t + 1\\
	y &= 2\sin t
\end{align}

\begin{align}
x &= t & x &= \cos t & x &= t\\
y &= 2t & y &= \sin (t+1) & y &= \sin t
\end{align}

\begin{align*}
	&(a+b)(a^2-ab+b^2) \\
={}& a^3-a^2b+ab^2+a^b
	 -ab^2+b^2\\
={}& a^3+b^3 \label{eq:cubenum}
\end{align*}

\begin{align*}
&\mathrel{\phantom{=}}
	(a+b)(a^2-ab+b^2) \\
&= a^3-a^2b+ab^2+a^b
	 -ab^2+b^2\\
&= a^3+b^3 % \label{eq:cubenum}
\end{align*}

\subsection{拆分多个公式}
multline环境、split环境\footnote{用在equation、gather环境}

\begin{multline}
a+b+c+d+e\\
+f+g+h+i+j\\
+k+l+m+n+o\\
+p+q+r+s+t
\end{multline}

\begin{equation}
\begin{split}
\cos 2x &= \cos^2 x - \sin^2 x\\
		&= 2\cos^2x - 1 
\end{split}
\end{equation}
\subsection{将公式组合成块}
cases,numcases环境,\&分隔值与条件

\begin{equation}
D(x)=
\begin{cases}
1,&\text{if }x \in \mathbf{Q};\\
0,&\text{if }x \in \mathbf{R}\setminus\mathbf{Q};
\end{cases}
\end{equation}

\[
\left.
\begin{gathered}
S \subseteq T\\
S \subseteq T
\end{gathered} \right\}
\implies S = T
\]

\[\text{比较曲线}
\left\{
\begin{lgathered}
x = \sin t, y = \cos t\\
x = t + \sin t, y = \cos t
\end{lgathered} 
\right.
\]




\section{精调与杂项}
\subsection{公式编号控制}
\subsection{公式的字号}
\subsection{断行与数学间距}

\end{document}